\documentclass[11pt]{amsart}%

\usepackage{a4wide}
\usepackage{amsfonts}
\usepackage{amsmath}
\usepackage{acronym}
\usepackage{amssymb}
\usepackage{graphicx}%
\setcounter{MaxMatrixCols}{30}
\usepackage{tabularx,bbm}
\usepackage{a4}
\usepackage[ngerman]{babel}
\usepackage[utf8]{inputenc}
\usepackage{amsthm}
\usepackage{amsthm,url}
\usepackage{paralist}
\usepackage{enumerate}
\usepackage{tabularx}
\usepackage{multirow}
\usepackage{stmaryrd}
\usepackage{color}
\usepackage{hyperref}


\def\0{\mathbf{0}}
\def\va{\mathbf{a}}
\def\e{\mathbf{e}}
\def\vf{\mathbf{f}}
\def\m{\mathbf{m}}
\def\s{\mathbf{s}}
\def\vu{\mathbf{u}}
\def\v{\mathbf{v}}
\def\vv{\mathbf{v}}
\def\bx{\mathbf{x}}
\def\vx{\mathbf{x}}
\def\z{\mathbf{z}}
\def\A{\mathbf{A}}
\def\K{\mathcal{K}}
\def\hs{h^*}
\def\hstar{h^*}
\def\lattice{\Lambda}
\def\suchthat{: \, }

\renewcommand\emptyset{\varnothing}
\newcommand\ds{\displaystyle}

\newcommand\fl[1]{\left\lfloor {#1} \right\rfloor} 
\newcommand\fr[1]{\left\{ {#1} \right\}} 
\newcommand\commentout[1]{}
\newcommand\Def[1]{{\bf #1}}
\newcommand\set[2]{{#1} : \, {#2}}

\newcommand{\todo}[1]{\textcolor{red}{\textbf{TODO:} #1}}

\renewcommand\vec{\overrightarrow}
\newcommand\intr{\operatorname{int}} 
\newcommand\rk{\operatorname{rank}} 
\newcommand\Asc{\operatorname{Asc}}
\newcommand\des{\operatorname{des}} 
\newcommand\Des{\operatorname{Des}} 
\newcommand\maj{\operatorname{maj}} 
\newcommand\lcm{\operatorname{lcm}} 
\newcommand\vol{\operatorname{vol}} 
\newcommand\conv{\operatorname{conv}} 
\newcommand\vspan{\operatorname{span}} 
\newcommand\verts{\operatorname{vert}} 
\newcommand\vertices{\operatorname{vert}} 
\newcommand\const{\operatorname{const}} 
\newcommand\cone{\operatorname{cone}} 
\newcommand\link{\operatorname{link}} 
\newcommand\height{\operatorname{height}} 
\newcommand\ehr{\operatorname{ehr}} 
\newcommand\Ehr{\operatorname{Ehr}} 
\newcommand\SL{\operatorname{SL}} 
\newcommand\GL{\operatorname{GL}} 

\newcommand\Z{\mathbb{Z}}
\newcommand\Q{\mathbb{Q}}
\newcommand\R{\mathbb{R}}
\newcommand\C{\mathbb{C}}
\newcommand\N{\mathbb{N}}

\setlength{\parindent}{0pt}
\setlength{\parskip}{4pt}

\pagestyle{empty}
\thispagestyle{empty}

\begin{document}

% \includegraphics[width=2in]{homework/HUJI_LogoEng_hor}
% \makebox[2in][l]{\Large HUJI logo ?}

\includegraphics[width=2in]{fu}

% \vspace{-.08in}
\small
{\sc Discrete Geometry I} \\ Wintersemester 2023/24
\normalsize

\vspace{-.9in}
\hspace{4.55in}
\begin{tabular}{r}
% \small Florian Frick\\
\small Prof. Giulia Codenotti\\
\small Sophie Rehberg\\
% \small Institut f\"ur Mathematik\\
% \small AG Diskrete Geometrie\\ 
% \small Arnimallee 2\\
% \small Freie Universit\"at Berlin\\
% \small {\tt beck@fu-berlin.de}
\end{tabular}

\def\uebung{1}

\newcounter{exerc}
\setcounter{exerc}{1}

\def\nextex{
\medskip
\noindent
\textbf{Exercise \uebung.\theexerc\ (5 points)}\\
\addtocounter{exerc}{1}
}

\def\nextexc{
\medskip
\noindent
\textbf{Exercise \uebung.\theexerc} (cooperative)\\
\addtocounter{exerc}{1}
}

\def\nextexi{
\medskip
\noindent
\textbf{Exercise \uebung.\theexerc} (individual)\\
\addtocounter{exerc}{1}
}

\begin{center}
\textbf{Organizational Information}\\
\end{center}

% \bigskip
\vspace{2\baselineskip}
\noindent
\textbf{Lectures}
\begin{itemize}
 \item Tuesdays 10-12 in T9/SR 006
 \item Wednesdays 10-12 in KöLu24-26/SR 006 
%  \item answer the corresponding quiz by Monday 11am 
%  \item join in person (rooms A3.024 @FUB, Ross70A @HUJI) % or online (link)
%  \item you will present and discuss two exercise solutions (see below)
% \item we will discuss typical exam questions 
%  \item we will discuss your questions regarding the videos, and
%    hint towards further topics in the field beyond the scope of this
%    course
\end{itemize}

\medskip
\noindent
\textbf{Exercise sessions}
\begin{itemize}
%  \item we meet online (see moodle for detailed meeting information or use this link: \url{https://huji.zoom.us/j/86357848003?pwd=d3N6WnoyMlJ6d3ROMVdwYTNjVFdvZz09}) on Wednesdays (08:30-10:00 Berlin
%    time, 09:30-11:00 Jerusalem time)
 \item Fridays, 10:15-11:45 in A6/SR 032
%  \item you have access to a physical room (A6.032 @FUB, ??? @HUJI)
 \item you present and discuss your solutions from previous exercise sheets
 \item you might discuss ideas on how to solve the upcoming exercise sheet
 \item you can ask questions about the lectures and exercises
 \item you might discuss further exercises
 \item {\footnotesize Sophie is there to answer your questions and help you}
\end{itemize}


\textbf{Final exams}
\begin{itemize}
	\item at the end of the course there will be a written exam
	\item first exam at 10:00 AM on February 20th
	% 2nd semester @ HUJI starts March 12
	\item second exam to be announced 
	% \item 
\end{itemize}

\textbf{Formal requirements}

To successfully complete the course you must do all the following:
\begin{itemize}
	\item pass the final exam
	 \item obtain active participation credit: that is, achieve at least 50 \% of the points on the homework sheets
	%   \begin{itemize}
	%    \item sign up for homework presentation  at least , 7 minutes! (regular attendence)
	%    \item and your group should have received at least 60\% of the available points of the graded exercises
	%   \end{itemize}
	\item obtain regular participation credit: that is, sign up  to present a homework problem during the exercise session at least every second week
\end{itemize}

\medskip
\noindent
\textbf{Homework}
\begin{itemize}
 \item due every week on Wednesdays 9:00 am
 \item submission in pairs
 \item every sheet contains 3 or 4 exercises, you submit written
   solutions to all the exercises 
%  \item during the lecture on Tuesdays, solutions of 
%    exercises are presented by students (sign-up via moodle)
   % fail once -- warning; fail twice -- fail.
 \item two exercises of your choice will be graded, sometimes exercises are marked as default graded.
%  \item each exercise gets either a ``reasonably well processed''-check or not


 \item homework submission will be through the Whiteboard, 
    your  homework submission file needs to fulfill all the following properties:
 \begin{enumerate}
  \item a readable pdf file not larger than 1MB,
  \item handwritten (readable) or properly typed in LaTeX,
  \item exercises in the correct ordering,
  \item name your file according to the following pattern:\\
   \url{hw##-lastname1_name1-lastname2_name2.pdf}
  \\(names of the group member ordered alphabetically with respect to
  the lastname), e.g., 
  \url{hw04-hirschhuegel_maria-rehberg_sophie.pdf}
 \end{enumerate}
 If your file does not comply with the above requirements it will not be graded!
 \item  
  We encourage you to work together within your homework group and across different groups, you are welcome to exchange ideas, discuss strategies, and solution drafts.
  Then write down the solution/proof in your own words and work out some missing details on your own. 
  
  If you copy entire solutions or proofs you will not get any points for the whole sheet and risk not passing the active participation at all.
  
%   You should NOT copy whole proofs or solutions (we will notice). 
\end{itemize}

\medskip
\noindent
\textbf{Communication}
\begin{itemize}
 \item material, announcements, homework etc. will be posted on the Whiteboard
%  \item please use the forum to ask (and answer?) questions (?)
 \item you can write mails: \url{s.rehberg@fu-berlin.de}
 \item there will be  a drop-by hour with Sophie, probably on Mondays 11:30 until 12:30 in Arnimallee 6, where the coffee machine is (to be confirmed).
 \item Prof. Codenotti has a weekly office hour on Wednesdays, 15.30-16:30.
%  \item 
 \item we encourage you to create your own communication channel to
   connect with your colleagues (Discord? Zulip?)
\end{itemize}

\medskip
\noindent



\medskip
\noindent
\textbf{Do you feel lost? Unorganized? Demotivated? Lonely?}
 Talk to us ! Otherwise, here are some more people you can talk to:
\begin{itemize}
 \item Student Advisory Service (Department of Mathematics and Computer Science
Logo Studienberatung), from students for students: \url{https://www.mi.fu-berlin.de/en/stud/beratungszentrum/studienberatung/index.html}
 \item Advising: Everything about your Studies at a Glance: \url{https://www.fu-berlin.de/en/studium/beratung/index.html}
 \item asta fu, students self-administration organization: \url{https://astafu.de/en}
 \item Center for Academic Advising and Psychological Counseling: \url{https://www.fu-berlin.de/en/sites/studienberatung/index.html}

%   \item \todo{}
%  \item
\end{itemize}


\end{document}

